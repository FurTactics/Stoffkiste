\documentclass[18pt] {article}
\usepackage[utf8]{inputenc}
\usepackage[margin=1in]{geometry}
\usepackage{framed}
\usepackage{titling}
\usepackage{amsmath}
\usepackage{amssymb}
\setlength{\droptitle}{-10em}
\linespread{1.2}
\title{Übersicht über die Algebraischen Strukturen aus der Vorlesung "Mathematische Methoden für Informatiker II"}
\date{\vspace{-4ex}}
\author{Klaus-Rudolf Kladny}
\begin{document}
\maketitle
\vspace{15mm}
\underline{\huge Gruppentheorie:}\\

\textbf{1. Halbgruppe $(S, *)$}
\begin{framed}
Eigenschaften: \\
\begin{enumerate}
\item Existenz einer Menge S (darf auch leer sein)

\item Existenz einer zweistelligen Operation:\\
$$*: S \times S \rightarrow S, (a, b) \mapsto a * b$$

\item Assoziativität bezüglich $*$ :\\
$$\forall a, b, c \in S:   a*(b*c) = (a*b)*c$$

\item Abgeschlossenheit bezüglich $*$ :\\
$$\forall a, b \in S:   (a*b) \in S$$
\end{enumerate}
\end{framed} 
\bigskip
\large{\underline{Unterhalbgruppe}}

Eine Unterhalbgruppe $(U, *)$ einer Halbgruppe $(S, *)$ ist eine Halbgruppe mit folgenden 

Eigenschaften:\\

\begin{enumerate}
\item $U \subseteq S$ und $U \neq \emptyset$

\item $a, b \in U \Rightarrow (a * b) \in U$
\end{enumerate}
\pagebreak
\textbf{2. Monoid $(S, *, e)$}
\begin{framed}
Eigenschaften: \\

\begin{enumerate}
\item Es gelten alle Eigenschaften für Halbgruppen

\item Existenz eines neutralen Elements $e$ :
$$\forall a \in S :   e*a = a*e = a$$
\end{enumerate}
\end{framed} 
\bigskip
\textbf{3. Gruppe $(S, *, e)$}
\begin{framed}
Eigenschaften: \\

\begin{enumerate}
\item Es gelten alle Eigenschaften für Monoide

\item Es existieren inverse Elemente:
\end{enumerate}
$$ \forall a \in S :  \exists! a^{-1} : a * a^{-1} = a^{-1} * a = e $$
\end{framed} 

\underline{Beispiele für Gruppen:}

Permutationsgruppen, Automorphismengruppen, Faktorgruppen, Menge der ganzen Zahlen 

mit der Addition $(\mathbb{Z}, +)$\\

\textbf{3.1 Abelsche Gruppe $(S, *, e)$}
\begin{framed}
Eigenschaften: \\
\begin{enumerate}
\item Es gelten alle Eigenschaften für Gruppen

\item Kommutativität bezüglich $*$ :\\
$$ \forall a, b \in S: a * b = b * a$$
\end{enumerate}
\end{framed} 
\pagebreak
\textbf{3.2 Zyklische Gruppe $(S, *, e)$}
\begin{framed}
Eigenschaften: \\

\begin{enumerate}
\item Es gelten alle Eigenschaften für Gruppen

\item Es existiert ein Element, welches mit $\cdot$ die gesamte Gruppe erzeugt. (erzeugendes Element)
\end{enumerate}
\end{framed} 

\underline{Bemerkung zu zyklischen Gruppen}: 

Sei im Algemeinen $n$ die Kardinalität der Gruppe
\begin{enumerate}
\item Es gibt zu jedem $n \in \mathbb{N}$ bis auf Isomorphie genau eine zyklische Gruppe mit dieser Kardinalität.
\item Die Gruppe hat $\phi (n)$ erzeugende Elemente.
\item Jede zyklische Gruppe ist abelsch (kommutativ bzgl. $\cdot$).
\item Das direkte Produkt zweier zyklischer Gruppen ist zyklisch, gdw. die Kardinalitäten der Gruppen teilerfremd sind.

\item Es gibt immer mindestens ein Element, welches mit $\cdot$ die gesamte Gruppe alleine erzeugt.

\item Zu jedem Teiler $t$ von $n$ existiert genau eine Untergruppe. Sei $a$ ein erzeugendes Element der gesamten Gruppe. Dann wird die Untergruppe der Ordung $t$ vom Element $a^{n/t}$ erzeugt.

\end{enumerate}

\large{\underline{Untergruppe}}

Eine Untergruppe $(U, *)$ einer Gruppe $(S, *)$ mit dem neutralen Element $e$ ist eine Gruppe 

mit folgenden Eigenschaften:\\

\begin{enumerate}
\item $U \subseteq S$ und $U \neq \emptyset$

\item $e \in U$

\item $a, b \in U \Rightarrow (a * b) \in U$

\item $a \in S \Rightarrow a^{-1} \in U$ (Muss nur in nicht endlichen Gruppen gezeigt werden)\\
\end{enumerate}
\pagebreak

\underline{Satz von Lagrange}:

$$\vert S  \vert = [S : U] \cdot \vert U \vert$$\\

Folgerung: Die Mächtigkeit jeder Untergruppe teilt die Mächtigkeit der Gruppe.\\


\bigskip
\underline{\huge Ringtheorie:}\\

\textbf{1. Halbring (auch: Semiring) $(H, +, \cdot)$}
\begin{framed}
Eigenschaften: \\

\begin{enumerate}
\item Existenz einer nichtleeren (!) Menge H

\item Existenz zweier zweistelliger Operationen:\\
$$+: H \times H \rightarrow H, \indent (a, b) \mapsto a + b$$
$$\cdot : H \times H \rightarrow H, \indent (a, b) \mapsto a \cdot b$$

\item $(H, +)$ ist eine kommutative Halbgruppe.

\item $(H, \cdot )$ ist eine Halbgruppe. 

\item Es gelten die Distributivgesetze:

$$ \forall a, b, c \in H : (a + b) \cdot c = a \cdot c + b \cdot c$$
$$ \forall a, b, c \in H : c \cdot (a + b) = c \cdot a + c \cdot b$$
\end{enumerate}
\end{framed}
\bigskip

\pagebreak
\textbf{2. Ring $(H, +, \cdot)$}
\begin{framed}
Eigenschaften: \\

\begin{enumerate}
\item Es gelten alle Eigenschaften für Halbringe.

\item (H, +) ist eine abelsche Gruppe.
\begin{enumerate}
\item Das neutrale Element wird als Nullelement $0$ bezeichnet
\end{enumerate}
\end{enumerate}
\end{framed} 
\bigskip


\textbf{3.1 kommutativer Ring $(H, +, \cdot)$}
\begin{framed}
Eigenschaften: \\

\begin{enumerate}
\item Es gelten alle Eigenschaften für Ringe

\item $(H, \cdot )$ ist eine kommutative Halbgruppe
\end{enumerate}

\end{framed} 
\bigskip
\textbf{3.2 euklidischer Ring $(H, +, \cdot)$}
\begin{framed}
Eigenschaften: \\

\begin{enumerate}

\item Es gelten alle Eigenschaften für Ringe

\item $\forall a, b \in H\setminus\{0\}:$ Es kann ein größter gemeinsamer Teiler $ggT(a, b)$ mit dem euklidischen Algorithmus bestimmt werden.
\end{enumerate}
\end{framed}
\pagebreak
\textbf{3.3 Integritätsring $(H, +, \cdot)$}
\begin{framed}
Eigenschaften: \\
\begin{enumerate}
\item Es gelten alle Eigenschaften für kommutative (!) Ringe.

\item Es gibt ein neutrales Element bezüglich der Multiplikation, welches als 

Einselement $1$ bezeichnet wird.

\item Es existieren keine Nullteiler:

(Erinnerung)$$ a \in H \ ist \ Nullteiler \Leftrightarrow \exists b \in H : a \cdot b = 0$$
\end{enumerate}
\end{framed}  

\textbf{3.3.1 Polynomring $(R[x], +, \cdot)$}
\begin{framed}
Eigenschaften: \\

\begin{enumerate}
\item $(R, +, \cdot)$ ist ein Ring.

\item Er ist ein nicht endlicher Integritätsring, also kein Körper.

\end{enumerate}
\end{framed}

\bigskip

\textbf{3.4 Körper $(H, +, \cdot)$}
\begin{framed}
Eigenschaften: \\

\begin{enumerate}
\item Es gelten alle Eigenschaften für kommutative (!) Ringe

\item Es gibt ein neutrales Element bezüglich der Multiplikation, welches als 

Einselement $1$ bezeichnet wird.

\item $(H, \cdot)$ ist eine Gruppe
\end{enumerate}
\end{framed} 
\bigskip
\pagebreak
\underline{\large Unterring}:

Ein Unterring $(U, +, \cdot)$ eines Rings $(H, +, \cdot)$ ist ein Ring mit folgenden Eigenschaften:\\

\begin{enumerate}
\item $U \subseteq S$ und $U \neq \emptyset$
\item
\begin{enumerate}
\item $a, b \in U \Rightarrow (a + b) \in U$

\item $a, b \in U\Rightarrow (a \cdot b) \in U$
\end{enumerate}

\item$a \in U \Rightarrow a^{-1} \in U$ (Dies muss nur in nicht endlichen Ringen gezeigt werden)
\end{enumerate}

\bigskip
\underline{Zusammenhang zwischen Integritätsring und Körper}:

Jeder Körper ist ein Integritätsring und jeder endliche Integritätsring ist ein Körper.\\

\textbf{3.4.1 Endlicher Körper (auch: Galoiskörper) $(H, +, \cdot)$}
\begin{framed}
Eigenschaften: \\

\begin{enumerate}

\item Es gelten alle Eigenschaften für Körper

\item Die Anzahl der Elemente ist endlich und sogar eine Primzahlpotenz
\end{enumerate}
\end{framed}
\bigskip
\underline{Multiplikative Gruppe}:

Ist die Gruppe, welche aus allen Elementen des Körpers mit Ausnahme des 

Nullelements mit $\cdot$ entsteht.\\
\bigskip

\underline{Bemerkung zu endlichen Körpern}:
\begin{enumerate}
\item Es gibt zu jeder Primpotenz bis auf Isomorphie genau einen endlichen Körper.

\item Es gibt immer ein primitives Element, welches mit $\cdot$ den gesamten Körper bis auf das Nullelement alleine erzeugt.

\end{enumerate}
\pagebreak
\textbf{3.4.1.1 Endlicher Polynomkörper  $(K[x]/f(x), +, \cdot)$}
\begin{framed}
Eigenschaften: \\
\begin{enumerate}
\item Es gelten alle Eigenschaften von endlichen Körpern.

\item $f(x)$ ist ein irreduzibles Polynom. Also $\nexists g(x), h(x) \in K[p]/f(x): g(x) \cdot h(x) = f(x)$\\

\end{enumerate}

Zusätzliche Eigenschaft:\\

Ist $x$ ein Erzeuger (\textit{Primitives Element} genannt) der multiplikativen Gruppe

$K[p]/f(x)\setminus\{0\}$, so wird $f(x)$ als \textit{Primitives Polynom} bezeichnet.
\end{framed}

\bigskip
\underline{Bemerkung zu endlichen Polynomkörpern}: \\

Sei im Allgemeinen $p$ die Primzahl und $k$ der Grad des irreduziblen Polynoms $f(x)$
\begin{enumerate}
\item Der Körper besteht aus $p^{k}$ Elementen.

\item Die multiplikative Gruppe des Körpers besteht aus $p^{k} - 1$ Elementen, da die $0$ nicht darin vorkommt.
\end{enumerate}
\end{document}
